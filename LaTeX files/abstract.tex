\chapter{Περίληψη}
\label{abstractname}
\noindent Τα συστήματα συστάσεων εντοπίζονται πλέον παντού στον κόσμο του διαδικτύου, καθορίζοντας απλές καθημερινές μας συνήθειες όπως την μουσική που θα ακούσουμε, τα προϊόντα που θα αγοράσουμε, τα βιβλία που θα διαβάσουμε και καταλήγουν να επηρεάζουν έμμεσα το τι σκεφτόμαστε και πως δρούμε ως πολίτες στην κοινωνία. Καθίσταται επομένως επιτακτική η ανάγκη ελέγχου αυτών των συστημάτων και η εύρεση της όποιας μεροληψίας εισάγουν. Το πιο γνωστό και σοβαρό είδος μεροληψίας είναι η μεροληψία δημοφιλίας (popularity bias). \\
Σε αυτή την διπλωματική εργασία αναφέρονται τα σημαντικότερα ζητήματα μεροληψίας και δικαιοσύνης που εντοπίζονται στην μηχανική μάθηση γενικότερα και στα συστήματα συστάσεων ειδικότερα. Στα πλαίσια αυτής της διπλωματικής εργασίας έχει αναπτυχθεί μια εφαρμογή η οποία επιτρέπει στους χρήστες να δημιουργήσουν ένα σύστημα συστάσεων, χρησιμοποιώντας ένα σύνολο δεδομένων της επιθυμίας τους, και ακολούθως να ελέγξουν εάν έχει εισαχθεί κάποια μεροληψία και να την μετριάσουν με χρήση ενός εκ των τεσσάρων αλγορίθμων που προσφέρονται: FAR, PFAR, FA*IR και Calibrated recommendations.\\
Με την εφαρμογή αυτή υλοποιήθηκαν πειράματα για τον εντοπισμό της μεροληψίας με χρήση τεσσάρων συνόλων δεδομένων, εκ των οποίων το ένα πραγματικό. Στη συνέχεια, γίνεται η αξιολόγηση των αποτελεσμάτων μέσω τριών διαφορετικών τύπων αναλύσεων: ανάλυση υπερπαραμέτρων των αλγορίθμων, σύγκριση αλγορίθμων και συνόλων δεδομένων και ανάλυση cut-off. Σε όλες τις αναλύσεις που πραγματοποιήθηκαν εξετάστηκε επίσης ο ρόλος των χαρακτηριστικών των δεδομένων, όπως η αραιότητα του μητρώου χρηστών αξιολογήσεων, ο λόγος αξιολογήσεων προς χρήστες, αξιολογήσεων προς αντικείμενα, χρηστών προς αντικείμενα και ο χώρος των αξιολογήσεων, δίνοντας ιδιαίτερη έμφαση στην αραιότητα των δεδομένων. Τέλος, γίνεται ο μετριασμός της μεροληψίας που εντοπίστηκε, με σύγκριση τριών διαφορετικών αλγορίθμων.\\
Από την ανάλυση που πραγματοποιήθηκε διαπιστώθηκε ότι σε όλα τα σύνολα δεδομένων, τα χαρακτηριστικά των δεδομένων επηρεάζουν έως έναν βαθμό την μεροληψία που εισάγεται. Παράλληλα, οι υπερπαράμετροι των αλγορίθμων παίζουν πολύ μεγάλο ρόλο στην ρύθμιση της μεροληψίας πέρα από την ρύθμιση την ακρίβειας. Μια ακόμη διαπίστωση που προέκυψε από την έρευνά μας είναι ότι οι post-processing αλγόριθμοι μετριασμού της μεροληψίας μπορούν να βελτιώσουν το αντιστάθμισμα μεροληψίας-ακρίβειας, ωστόσο έχουν και σημαντικούς περιορισμούς. Εν κατακλείδι, οι δημιουργοί των συστημάτων είναι αναγκαίο αφενός να έχουν επίγνωση της μεροληψίας που εισάγεται, καθώς και των αιτιών της, και αφετέρου θα πρέπει να φροντίζουν να βρίσκουν ένα αντιστάθμισμα ανάμεσα στην ακρίβεια και την μεροληψία. Αυτό μπορεί να συμβεί είτε με την κατάλληλη ρύθμιση των υπερπαραμέτρων είτε με τον μετριασμό της μεροληψίας. H εφαρμογή που αναπτύχθηκε στα πλαίσια αυτής της διπλωματικής εργασίας συμβάλλει σημαντικά προς αυτή την κατεύθυνση.

\vspace*{5mm}
\noindent{\LARGE \textbf{Λέξεις κλειδιά}}\\
Μηχανική μάθηση, Συστήματα συστάσεων, Μεροληψία δημοφιλίας, Δικαιοσύνη
%1. Reason for writing:
%What is the importance of the research? Why would a
%reader be interested in the larger work?
%2. Problem:
%What problem does this work attempt to solve? What is
%the scope of the project? What is the main argument,
%thesis or claim?
%3. Methodology:
%An abstract of a scientific work may include specific
%models or approaches used in the larger study. Other
%abstracts may describe the types of evidence used in
%the research.
%4. Results:
%An abstract of a scientific work may include specific data
%that indicates the results of the project. Other abstracts
%may discuss the findings in a more general way.
%5. Implications:
%How does this work add to the body of knowledge on
%the topic? Are there any practical or theoretical
%applications from your findings or implications for future
%research?

