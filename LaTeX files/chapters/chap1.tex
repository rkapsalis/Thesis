\chapter{Εισαγωγή}
\label{chap1}


\section{Αντικείμενο της διπλωματικής}

\noindent Η ραγδαία ανάπτυξη των αλγορίθμων τεχνητής νοημοσύνης την τελευταία δεκαετία, έχει οδηγήσει πολλές εταιρείες, κυβερνήσεις και οργανισμούς να υιοθετήσουν τέτοιες λύσεις και να επωφεληθούν από αυτές. Οι χρήσεις της τεχνητής νοημοσύνης εκτείνονται πλέον σε ένα πολύ μεγάλο εύρος από τα ρομποτικά συστήματα, τη βιομηχανία και τον τραπεζικό τομέα, έως την βιολογία, την γεωργία και την αρχαιολογία. Το γεγονός όμως ότι πλέον ορισμένοι αλγόριθμοι επηρεάζουν τις ζωές μας και την καθημερινότητά μας, είτε έμμεσα είτε άμεσα, έχει φέρει στο προσκήνιο μια σειρά συζητήσεων για τα ηθικά ζητήματα που προκύπτουν από τη χρήση τους. Στην παρούσα εργασία θα μας απασχολήσουν κυρίως οι αλγόριθμοι μηχανικής μάθησης, ενός πεδίου της τεχνητής νοημοσύνης. Τα τελευταία χρόνια η επιστημονική κοινότητα δείχνει ιδιαίτερο ενδιαφέρον για ζητήματα ηθικής όπως η δικαιοσύνη (fairness), η μεροληψία (bias) και η ανάγκη για λογοδοσία (accountability), διαφάνεια (transparency), επεξηγησιμότητα (explainability) και ερμηνευσιμότητα (interpretability) στη μηχανική μάθηση. Η περισσότερη έρευνα φαίνεται πως έχει γίνει για την κατηγοριοποίηση (classification), κυρίως για το binary classification, και για την παλινδρόμηση (regression). Σε αυτή την κατεύθυνση έχουν δημιουργηθεί διάφορα εργαλεία ανοικτού κώδικα (open-source), με το πιο γνωστό και ίσως πιο πλήρες να είναι το εργαλείο AIF360, το οποίο περιέχει διάφορες μετρικές για το fairness και το bias, επεξηγήσεις τους και  αλγορίθμους μετριασμού της μεροληψίας. Ωστόσο, στα συστήματα συστάσεων δεν φαίνεται να έχει γίνει ανάλογη προσπάθεια, τουλάχιστον όχι σε τέτοιο βαθμό, ενώ δεν υπάρχει διαθέσιμο κάποιο παρόμοιο εργαλείο.\\
 Έναυσμα για την ενασχόληση με το συγκεκριμένο θέμα αποτέλεσε το άρθρο με τίτλο ``How Facebook got addicted to spreading misinformation" της Karen Hao \cite{haoHowFacebookGot} που δημοσιεύθηκε στο MIT Technology Review, ενός διμηνιαίου περιοδικού με ποικίλα θέματα επιστήμης και τεχνολογίας που εκδίδεται από το φημισμένο Πανεπιστήμιο των Η.Π.Α. Massachusetts Institute of Technology (Μ.Ι.Τ.). Στο συγκεκριμένο άρθρο, γίνεται εκτενής αναφορά στο πως οι αλγόριθμοι των συστημάτων συστάσεων που χρησιμοποιούνται από τον τεχνολογικό κολοσσό Facebook με στόχο την αύξηση της αφοσίωσης (engagement) έχουν ολέθρια αποτελέσματα στην κοινωνία. Προκειμένου να αντιληφθούμε το μέγεθος του κακού που έχει προκληθεί στην κοινωνία, αρκεί να αναφέρουμε πως η προώθηση όλο και πιο αρνητικού υλικού που κινδύνευε να επιδεινώσει περαιτέρω την ψυχική υγεία ατόμων που ήταν αρκετά ευάλωτοι ψυχολογικά, η παραπληροφόρηση και ο εξτρεμισμός που οδήγησαν σε έναν εμφύλιο πόλεμο στην Μιανμάρ, επηρέασαν εκλογικές αναμετρήσεις και κατεύθυναν πολλούς ανθρώπους στο να μην κάνουν το εμβόλιο στον καιρό της πανδημίας COVID-19 μέσα από την συστηματική προώθηση ψευδών ειδήσεων, είναι ορισμένα από όσα αναφέρει το άρθρο. Επομένως, όπως γίνεται αντιληπτό καθίσταται επιτακτική η ανάγκη δημιουργίας ενός εργαλείου για τον εντοπισμό και τον έλεγχο της μεροληψίας στα συστήματα συστάσεων.


\subsection{Συνεισφορά}

\noindent Η συνεισφορά της διπλωματικής συνοψίζεται ως εξής:
\begin{enumerate}
	\item δημιουργία διαδικτυακής εφαρμογής στην οποία ένας χρήστης μπορεί να: \begin{itemize}
		\item αναλύσει ένα σύνολο δεδομένων
		\item να δημιουργήσει συστάσεις προς τους χρήστες επιλέγοντας μέσα από μια πληθώρα αλγορίθμων και ρυθμίζοντας κατάλληλα τις υπερπαραμέτρους αυτών εύκολα και γρήγορα
		\item να αξιολογήσει τα αποτελέσματα ως προς την ακρίβεια, το diversity, το novelty, το popula\-rity bias και την κάλυψη των αντικειμένων επιλέγοντας τις μετρικές αξιολόγησης που επιθυμεί
		\item να μετριάσει την μεροληψία χρησιμοποιώντας 4 διαφορετικούς αλγορίθμους και στη συνέχεια να δει σε τι ποσοστό έχει υπάρξει βελτίωση
		\item να προβάλλει σε γλώσσα φιλική προς όλους τους χρήστες ανεξαρτήτως των γνώσεών τους, επεξηγήσεις για όλες τις μετρικές αξιολόγησης 
	\end{itemize}
		\item	υλοποίηση πειράματος, μέσω της ανωτέρω εφαρμογής, χρησιμοποιώντας ένα πραγματικό σύνολο δεδομένων και τρία σύνολα δεδομένων που συλλέξαμε από το διαδίκτυο, και δημιουργία συστημάτων συστάσεων με χρήση 11 διαφορετικών αλγορίθμων από 5 διαφορετικές οικογένειες. Στη συνέχεια αξιολόγηση αυτών των συστημάτων ως προς την ακρίβεια και την μεροληψία που (ενδεχομένως) εισάγουν.
		\item	διερεύνηση του κατά πόσο οι υπερπαράμετροι των αλγόριθμων συστάσεων επηρεάζουν την ακρίβεια και το popularity bias και πως μπορεί να βρεθεί ένα αντιστάθμισμα ανάμεσα σε αυτά τα δύο.
		\item	σύγκριση των αλγορίθμων και των συνόλων δεδομένων και αναζήτηση του κατά πόσο το popula\-rity bias και η ακρίβεια επηρεάζονται από τα χαρακτηριστικά των δεδομένων.
		\item	σύγκριση δύο διαφορετικών τεχνικών μετριασμού της μεροληψίας και σύγκριση τριών αλγορίθμων που ανήκουν στην ίδια κατηγορία.

\end{enumerate}


\section{Διάρθρωση της διπλωματικής εργασίας}
\noindent Η παρούσα εργασία αποτελείται από πέντε κεφάλαια.
\begin{itemize}
	\item  Στο πρώτο κεφάλαιο γίνεται η παρουσίαση βασικών εννοιών της μηχανικής μάθησης και των συστημάτων συστάσεων.
	\item Στο δεύτερο κεφάλαιο θίγονται αρχικά ορισμένα ζητήματα που σχετίζονται με την ηθική στην μηχανική μάθηση όπως η αμεροληψία, η δικαιοσύνη, η επεξηγησιμότητα και η διαφάνεια. Στη συνέχεια του κεφαλαίου οι παραπάνω έννοιες επεκτείνονται και αναλύονται πιο ειδικά, για τα συστήματα συστάσεων.
	\item Στο τρίτο κεφάλαιο γίνεται η παρουσίαση της εφαρμογής που υλοποιήθηκε στα πλαίσια της διπλωματικής εργασίας. 
	\item Στο τέταρτο κεφάλαιο γίνεται η ανάλυση της πειραματικής αξιολόγησης, παρουσιάζονται αναλυτικά τα σύνολα δεδομένων, οι αλγόριθμοι και οι μετρικές αξιολόγησης που χρησιμοποιήθηκαν γίνεται η παρουσίαση και η ανάλυση των αποτελεσμάτων όπως αυτά προέκυψαν από τα πειράματα που διεξήχθησαν, ενώ παρουσιάζονται και συγκρίνονται τρεις διαφορετικοί αλγόριθμοι μετριασμού της μεροληψίας.
	\item Στο πέμπτο κεφάλαιο παρουσιάζονται τα τελικά συμπεράσματα, οι περιορισμοί που συναντήθηκαν και μελλοντικές προεκτάσεις.
\end{itemize}