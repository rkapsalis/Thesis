\chapter{Συμπεράσματα και μελλοντικές προεκτάσεις}
\label{chap6}
\section{Συμπεράσματα}
\noindent Στη βιβλιογραφία τα τελευταία χρόνια γίνεται μεγάλη συζήτηση για τα ηθικά ζητήματα στη μηχανική μάθηση. Ωστόσο, η έρευνα που έχει γίνει στα συστήματα συστάσεων δεν είναι ανάλογη με αυτή που έχει γίνει σε άλλους τομείς, όπως στην κατηγοριοποίηση, την υπολογιστική όραση (computer vision) και την παλινδρόμηση. Λαμβάνοντας υπόψη μας αυτό, αυτή η διπλωματική στοχεύει στο να:
 \begin{itemize}
 	\item συμβάλλει στην έρευνα γύρω από αυτό το ζήτημα
 	\item βοηθήσει τους ερευνητές, τους διαχειριστές συστημάτων και τους χρήστες  - ακόμη και εκείνους χωρίς ιδιαίτερες τεχνικές γνώσεις - , να εντοπίσουν και να μετριάσουν την μεροληψία ενός συστήματος συστάσεων, παρέχοντάς τους ταυτόχρονα την δυνατότητα να το δημιουργήσουν, όπως εκείνοι επιθυμούν, ενισχύοντας την διαφάνεια και την ελεγξιμότητα.
 	\item αποτελέσει μια από τις πρώτες προσπάθειες ελέγχου της μεροληψίας σε ένα πραγματικό σύνολο δεδομένων
 	\item γνωστοποιήσει στο ευρύ κοινό ζητήματα μεροληψίας στα συστήματα συστάσεων, δίνοντας ιδιαίτερη έμφαση στο popularity bias.
 \end{itemize} 
Στα πλαίσια αυτής της εργασίας αναπτύχθηκε μια διαδικτυακή εφαρμογή η οποία αφενός αποτέλεσε το εργαλείο για τη δημιουργία και την προβολή των αναλύσεων και των συγκρίσεων του πειράματος που διεξήχθη στα πλαίσια αυτής της εργασίας και αφετέρου αποτελεί ένα πολύτιμο εργαλείο για τον απλό χρήστη, τον ερευνητή και τον διαχειριστή ενός συστήματος. Η εφαρμογή αυτή, παρέχει στον χρήστη τις ακόλουθες δυνατότητες.\\
Αρχικά του επιτρέπει να μεταφορτώσει ένα σύνολο δεδομένων της αρεσκείας του και να ελέγξει για τυχόν ύπαρξη μεροληψίας σε αυτό μέσω αρκετά εύκολα κατανοητών γραφημάτων και πινάκων. Στη συνέχεια, να δημιουργήσει τα συστήματα συστάσεων όπως εκείνος επιθυμεί επιλέγοντας μέσα από μια πληθώρα αλγορίθμων και με δυνατότητα να ρυθμίσει κατάλληλα τις υπερ-παραμέτρους τους. Σε επόμενο βήμα, και αφού ο χρήστης έχει δημιουργήσει το σύστημα συστάσεων, μπορεί να ελέγξει για τυχόν ύπαρξη popularity bias, για το πόσα αντικείμενα καλύπτονται, για το diversity και το novelty. Ο έλεγχος αυτός περιλαμβάνει την ανάλυση υπερ-παραμέτρων, τη σύγκριση διαφορετικών συνόλων δεδομένων μεταξύ τους και την σύγκριση των αποτελεσμάτων για διαφορετικές τιμές μεγέθους λιστών συστάσεων ανά χρήστη. Ενώ τέλος, παρέχει και τη δυνατότητα για μετριασμό ενδεχόμενης μεροληψίας που εντοπίστηκε στο προηγούμενο βήμα, μέσω 4 διαφορετικών αλγορίθμων (FAR, PFAR, Calibrated recommendations, FA*IR) που όλοι ανήκουν στην κατηγορία των post-processing αλγορίθμων.
Στην εφαρμογή επίσης παρέχονται εξηγήσεις για κάθε διαθέσιμη μετρική αξιολόγησης σε γλώσσα απλή και κατανοητή για όλους τους χρήστες.\\
Αξίζει να σημειωθεί, πως με την χρήση της εφαρμογής ο χρήστης έχει άμεση πρόσβαση στα δύο κύρια σημεία εισαγωγής της μεροληψίας: στα δεδομένα και στον αλγόριθμο.

\noindent Με την ολοκλήρωση της ανάπτυξης της εφαρμογής διεξήχθησαν μέσω αυτής ορισμένα πειράματα που σχετίζονται με την εύρεση και τον μετριασμό της μεροληψίας των συστημάτων συστάσεων σε τέσσερα διαφορετικά σύνολα δεδομένων, ένα εκ των οποίων παραχωρήθηκε από μεγάλη εταιρεία λιανικών πωλήσεων. Αρχικά δημιουργήθηκαν συστήματα συστάσεων με χρήση 11 αλγορίθμων από 4 διαφορετικές οικογένειες αλγορίθμων. Η αξιολόγηση των συστημάτων συστάσεων έγινε με τη χρήση 12 διαφορετικών μετρικών κάλυψης αντικειμένων, ακρίβειας, μεροληψίας δημοφιλίας, diversity και novelty. 
Στο πρώτο μέρος του πειράματος εξετάστηκε το κατά εάν και πόσο επηρεάζουν οι υπερ-παράμετροι κάθε αλγορίθμου την μεροληψία και την ακρίβεια. Διαπιστώθηκε πως σε όλους τους αλγορίθμους\textit{ η ρύθμιση των υπερ-παραμέτρων επηρεάζει πολύ το ποσοστό εισαγωγής της μεροληψίας} και μάλιστα επειδή στην πλειονότητα των περιπτώσεων η ακρίβεια είναι αντιστρόφως ανάλογη της μεροληψίας, μπορεί αρκετά εύκολα να εισαχθεί αρκετή μεροληψία εάν κατά την ρύθμιση των υπερπαραμέτρων μας ενδιαφέρει περισσότερο η ακρίβεια ή ακόμη χειρότερα αγνοήσουμε τελείως την μεροληψία. Ένα ακόμη μέρος του πειράματος αποτέλεσε η σύγκριση των συνόλων δεδομένων. Η αραιότητα πέρα από την ακρίβεια επηρεάζει αρκετά, σε σχεδόν όλες τις περιπτώσεις, το novelty και το diversity, ενώ σε πολλές περιπτώσεις επιδρά αρνητικά και στο popularity bias. Το συμπέρασμα που προέκυψε εδώ είναι πως \textit{εκτός από την αραιότητα των δεδομένων, που έχει αναφερθεί και εκτενώς στη βιβλιογραφία, ένας αρκετά σημαντικός παράγοντας για τον οποίο δεν έχει γίνει μεγάλη αναφορά στην βιβλιογραφία είναι τα χαρακτηριστικά των δεδομένων,} καθώς παρατηρήθηκε το φαινόμενο σύνολα δεδομένων με παρόμοια αραιότητα να έχουν αρκετά διαφορετική συμπεριφορά. Μάλιστα στην βιβλιογραφία δεν κατορθώσαμε να βρούμε κάτι που να εξετάζει και τα χαρακτηριστικά των δεδομένων και τις υπερπαραμέτρους. \\
Θα πρέπει να τονίσουμε πως παρόμοια συμπεράσματα προέκυψαν και για το πραγματικό σύνολο δεδομένων που χρησιμοποιήσαμε, όπου τα δημοφιλή προϊόντα είναι λίγα. Αυτό το φαινόμενο όπως γίνεται εύκολα κατανοητό, έχει επιπτώσεις τόσο στους χρήστες, όσο και στην εταιρία και (έμμεσα) στους κατασκευαστές των προϊόντων.
Επιπρόσθετα, η αραιότητα και τα χαρακτηριστικά των δεδομένων μπορούν να επηρεάσουν πάρα πολύ την ακρίβεια των αλγορίθμων και επομένως να δίνονται προτάσεις στους χρήστες τελείως άσχετες με τα ενδιαφέροντά τους και το προφίλ τους.\\
Στο \cite{abdollahpouriManagingPopularityBias2019a} γίνεται μια αναφορά στο το πως η αραιότητα των δεδομένων επηρεάζει και το popularity bias, πέρα από την ακρίβεια, χωρίς ωστόσο το ζήτημα να εξετάζεται εκτενώς.
Επιπρόσθετα, κάτι που παρατηρήθηκε κατά την ανάλυσή μας είναι πως \textit{υπάρχουν αλγόριθμοι οι οποίοι εισάγουν (πολύ) περισσότερη μεροληψία σε σχέση με άλλους}, όπως ο BPRMF, ο SVD++ και άλλοι όπως οι NGCF και ItemKNN που είναι πιο δίκαιοι στις προβλέψεις τους, κάτι που έρχεται σε πλήρη συμφωνία με διάφορες μελέτες που έχουν γίνει. Ωστόσο, για όσους έχουμε καλά αποτελέσματα θα πρέπει να γίνει ρύθμιση υπερπαραμέτρων, προκειμένου να βρεθεί μια ισορροπία με την ακρίβεια.\\
Δοκιμάσαμε επίσης συνολικά τέσσερις αλγορίθμους μετριασμού οι οποίοι ανήκουν σε δύο κατηγορίες τεχνικών. Τρεις που ανήκουν στην κατηγορία του post-processing που εφαρμόζουν την τεχνική του re-ranking και έναν που ανήκει στην κατηγορία του in-processing. Στην γενική περίπτωση στην post-processing κατηγορία ο αλγόριθμος Cali πετυχαίνει την καλύτερη μείωση μεροληψίας, ωστόσο καλύτερο αντιστάθμισμα μεροληψίας-ακρίβειας επιτυγχάνει ο αλγόριθμος FAR. \textit{Η κατηγορία post-processing αποδείχθηκε μέσω των πειραμάτων ότι δεν ενδείκνυται για σύνολα δεδομένων με πολύ μεγάλη αραιότητα, ενώ δεν είναι αρκετά αποτελεσματική σε περίπτωση όπου έχουμε πολύ μεγάλη εισαγωγή μεροληψίας}, κάτι που είναι γνωστό και από την βιβλιογραφία. Σε αυτή την περίπτωση θα πρέπει να χρησιμοποιήσουμε μια in-processing τεχνική είτε συνδυασμό τεχνικών (στις περισσότερες περιπτώσεις).\\

\noindent Ύστερα και από έρευνα που διεξήγαμε δεν βρήκαμε κάτι παρόμοιο με την εφαρμογή που υλοποιήθηκε στα πλαίσια της παρούσας εργασίας. Ενώ αξίζει να σημειωθεί αποτελεί μια από τις πρώτες προσπάθειες διερεύνησης του φαινομένου της εισαγωγής μεροληψίας στα συστήματα συστάσεων που χρησιμοποιεί ένα πραγματικό σύνολο δεδομένων (έστω και αν αυτό έχει τροποποιηθεί ελαφρώς).
Κλείνοντας, η συνεισφορά της εργασίας σε κοινωνικό και ερευνητικό επίπεδο και στον επιχειρηματικό τομέα κρίνουμε πως είναι η εξής:\begin{itemize}
	\item Στον τομέα του ηλεκτρονικού εμπορίου αποδείχθηκε,  ύστερα και από την εξέταση ενός πραγματικού συνόλου δεδομένων,  πως το φαινόμενο της μεροληψίας δημοφιλίας είναι αρκετά έντονο, ενώ αρκετά αντικείμενα μπορεί να μην υπάρχουν σε καμία λίστα συστάσεων που δίνονται από τους αλγορίθμους στους χρήστες, μειώνοντας κατά πολύ τις πιθανότητες να τα δουν οι χρήστες και επομένως να πωληθούν. 
	\item Σε κοινωνικό επίπεδο, η εφαρμογή που αναπτύχθηκε αποτελεί ένα εργαλείο για τη δημιουργία ενός συστήματος συστάσεων, εύκολα και γρήγορα για εύκολο και γρήγορο έλεγχο των συστημάτων συστάσεων και μετριασμό του popularity bias σε κάθε σύνολο δεδομένων που περιέχει explicit δεδομένα ανεξαρτήτως τομέα. 
	\item Σε επιστημονικό επίπεδο, η εφαρμογή μπορεί να χρησιμοποιηθεί για δημιουργία πειραμάτων στα συστήματα συστάσεων είτε αυτά αφορούν ηθικά ζητήματα, όπως η μεροληψία και η δικαιοσύνη, είτε όχι.
\end{itemize}
\section{Περιορισμοί} \label{limitations}
\noindent Αρχικά, στο διαδίκτυο υπάρχουν πολύ λίγα σύνολα δεδομένων συστημάτων συστάσεων, που να πληρούν τα κριτήρια που είχαμε θέσει. Μια ακόμη δυσκολία που κληθήκαμε να αντιμετωπίσουμε είναι πως στο πραγματικό σύνολο δεδομένων δεν υπήρχε καμία πληροφορία σχετική με τους χρήστες και αναγκαστήκαμε να δημιουργήσουμε τυχαίους χρήστες. Όπως γίνεται αντιληπτό, αυτό αποτέλεσε τροχοπέδη σε πολλές περιπτώσεις, καθώς μπορεί -άθελά μας- να εισάγαμε κάποιο είδος μεροληψίας, κάτι που θα έκανε τα συμπεράσματα που θα προέκυπταν από το πείραμα ιδιαίτερα επισφαλή, αναγκάζοντάς μας έτσι να μην τα χρησιμοποιήσουμε σε ορισμένα μέρη του πειράματος.
\section{Μελλοντικές προεκτάσεις}
\noindent Όσον αφορά την εφαρμογή που υλοποιήθηκε, ως επέκτασή της στο μέλλον σχεδιάζεται να προστεθούν οι εξής λειτουργίες:\begin{enumerate}
	\item παροχή της δυνατότητας στον χρήστη να μπορεί να μεταφορτώνει τα δικά του αρχεία αποτελεσμάτων
	\item δυνατότητα δημιουργίας λογαριασμού για κάθε χρήστη, δημιουργώντας ειδική σελίδα για την εγγραφή και την σύνδεση του χρήστη.
	\item προσθήκη περισσότερων αλγορίθμων μετριασμού της μεροληψίας που να ανήκουν σε διαφορετικές κατηγορίες.
	\item βελτίωση του χρόνου υπολογισμού μέσω διάφορων τεχνικών (μέσω παραλληλοποίησης, αποθήκευσης στην cache ή οποιασδήποτε άλλης τεχνικής.).
\end{enumerate}
 Τέλος, όσον αφορά τα πειράματα που υλοποιήθηκαν θα μπορούσε να γίνει και χρήση implicit αξιολογήσεων και σύγκριση με τα explicit ως προς την εισαγωγή της μεροληψίας και χρήση περισσότερων συνόλων δεδομένων.\\\\
\noindent Εν κατακλείδι, οι αλγόριθμοι τεχνητής νοημοσύνης έχουν εισέλθει πλέον σε πολλές πτυχές της καθημερινότητάς μας. Ειδικότερα, οι αλγόριθμοι συστημάτων συστάσεων μπορεί να εισάγουν ή να τροποποιηθούν κατάλληλα για να εισάγουν μεροληψία, η οποία αποσαθρώνει την κοινωνία και αποτελεί σοβαρό κίνδυνο για την δημοκρατία, αφαιρώντας τη δυνατότητα κριτικής σκέψης από τον πολίτη μιας χώρας, τον πελάτη ενός ηλεκτρονικού καταστήματος και τον καταναλωτή ψυχαγωγικού περιεχόμενου. Ελπίζουμε η εργασία αυτή να αποτελέσει εφαλτήριο για περισσότερη και πιο ενδελεχή έρευνα στα συστήματα συστάσεων και να βοηθήσει όλους τους χρήστες ανεξαρτήτως των γνώσεών τους, να εντοπίσουν αλλά και να μετριάσουν την μεροληψία στα συστήματα συστάσεων.
